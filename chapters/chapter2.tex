\chapter{Table, Image, Caption, Label}
\section{Inserting table}
\begin{table}[h]
    \centering
    \begin{tabular}{|c|c|c|}
        \hline
        Column 1 & Column 2 & Column 3\\
        \hline
        Value 1 & Value 2 & Value 3 \\
        \hline
        Value 4 & Value 5 & Value 6 \\
        \hline
    \end{tabular}
    \caption{A sample table}
    \label{tab:sample1}
\end{table}

\begin{table}[h!]
        \centering
        \begin{tabular}{|c|p{10cm}|}
            \hline
            \textbf{Parameter} & \textbf{Position} \\
            \hline
            h & Place the float here, i.e., approximately at the same point it occurs in the source text (however, not exactly at the spot)\\
            \hline
            t & Position at the top of the page.\\
            \hline
            b & Position at the bottom of the page.\\
            \hline
            p & Put on a special page for floats only.\\
            \hline
            ! & Override internal parameters LaTeX uses for determining "good" float positions.\\
            \hline
            H & Places the float at precisely the location in the LaTeX code. Requires the float package, though may cause problems occasionally. This is somewhat equivalent to h!. \\
            \hline
        \end{tabular}
        \caption{Different positioning values}
        \label{tab:Positioning}
\end{table}
\begin{table}[h!]
    \centering
    \begin{tabular}{|c|p{10cm}|}
        \hline
        \textbf{Abbreviation} & \textbf{Value} \\
        \hline
        pt & a point is approximately 1/72.27 inch, that means about 0.0138 inch or 0.3515 mm (exactly point is defined as 1/864 of American printer’s foot that is 249/250 of English foot) \\
        \hline
        mm & a millimeter \\
        \hline
        cm & a centimeter \\
        \hline
        in & inch \\
        \hline
        ex & roughly the height of an 'x' (lowercase) in the current font (it depends on the font used) \\
        \hline
        em & roughly the width of an 'M' (uppercase) in the current font (it depends on the font used) \\
        \hline
        mu & math unit equal to 1/18 em, where em is taken from the math symbols family \\
        \hline
        sp & so-called "special points", a low-level unit of measure where 65536sp=1pt \\
        \hline
    \end{tabular}
    \caption{\LaTeX{} Units}
    \label{tab:Units}
\end{table}

\newpage
\section{Inserting Image}
\begin{figure}[h!]
  \centering
  \includegraphics[width=0.5\textwidth]{logo.jpg}
  \caption{Sample image}
  \label{fig:sample}
\end{figure}

\section{Labels}
Referencing Figure \ref{fig:sample}.

Referencing Table \ref{tab:sample1}.

Referencing Positioning Values \ref{tab:Positioning}.

\newpage
\section{Writing Mathematical Functions}

\[
f(x) = ax^2 + bx + c
\]

Where \(a\), \(b\), and \(c\) are constants.\\\\

Inline: \( E = mc^2 \) or $ E = mc^2 $ or \begin{math} E = mc^2 \end{math}\\

Display Math Mode (Unnumbered):
\begin{displaymath}
    E = mc^2
\end{displaymath}

or

\[E = mc^2\]

Display Math Mode (Equation Number):
\begin{equation}
    \label{eq:relativity}
    E = mc^2
\end{equation}



\begin{equation} 
    \label{eq:gravitation}
    F = G \frac{m_1 m_2}{r^2}
\end{equation}
where:
\begin{itemize}
    \item \( F \) is the force between two masses \( m_1 \) and \( m_2 \),
    \item \( G \) is the gravitational constant,
    \item \( r \) is the distance between the masses.
\end{itemize}

Referencing equation relativity: \ref{eq:relativity}\\
Referencing equation gravitation: \ref{eq:gravitation}
